% ==================================================
% 1S - DERIVATION - EXERCICE 2 - Calcul Des Derivees
% ==================================================

% ADD BEAMER THEME
\documentclass[t]{beamer}

% ADD PACKAGES
\usepackage{etex}
\usepackage[english]{babel}
\usepackage[utf8]{inputenc}
\usepackage{pstricks-add}
\usepackage{pst-eucl}

\beamertemplatenavigationsymbolsempty

% DEFINE COLORS
\definecolor{red}{HTML}{DD0000}
\definecolor{green}{RGB}{51,110,23}
\setbeamercolor{block title example}{fg=green}
	
% DOCUMENT
\begin{document}

	\begin{frame}[label=pagebanale]
		\frametitle{Dérivation - Exercice 1 : Nombre Dérivé}
		\pause
		\underline{Dans l'exercice 1, il y avait cette consigne :}

		\pause
		Déterminer les équations des tangentes aux courbes des fonctions suivantes :
		\pause
		\begin{enumerate}[a)]
			\item<+-> f définie sur $\mathbb{R}$ par $f(x) = 3x^3$ (au point d'abscisse 2)
			\item<+-> g définie sur $[-1;+\infty ]$ par $g(x) = \sqrt{x+1}$ (au point d'abscisse 5)
			\item<+-> i définie sur $\mathbb{R}^* (\textcolor{green}{]-\infty;0[\cup]0;+\infty[})$ par $i(x) = \dfrac{1}{x}$ (au point d'abscisse 3)
			\item<+-> j définie sur $\mathbb{R}$ par $j(x) = x^2 - 5x +7$ (au point d'abscisse -2)
		\end{enumerate}
		\pause
		Avec le cours précédent, \pause nous avons une autre manière de calculer la dérivée d'une fonction. \pause Nous n'allons plus nous embêter à calculer la limite du taux d'accroissement lorsque $h \rightarrow 0$.

		\pause Nous allons directement déterminer \textcolor{red}{la fonction dérivée}, \pause grâce aux \textcolor{green}{3 fonctions usuelles} \pause et aux \textcolor{green}{5 règles de dérivation}. 
	\end{frame}

	\begin{frame}
		\frametitle{a) $f(x) = 3x^3$ au point d'abscisse 2}
		\pause
		\begin{block}{}
			Pour déterminer l'équation de la tangente d'une fonction, vous devez (\underline{maintenant}) :
			\pause
			\begin{enumerate}[1.]
				\item Déterminer la fonction dérivée f'(x) de f(x)
				\pause
				\item Calculer f'(a)
				\pause
				\item Puis finalement, déterminer l'équation de la tangente
			\end{enumerate}
		\end{block}
	\end{frame}

	\begin{frame}
		\frametitle{a) $f(x) = 3x^3$ au point d'abscisse 2}
		\pause
		\begin{block}{1) Déterminer la fonction dérivée f'(x) de f(x)}
			\pause
			\begin{enumerate}[]
				\pause
				\item f(x) est une fonction polynôme définie sur $\mathbb{R}$, \pause donc elle est dérivable sur $\mathbb{R}$.
				\pause
				\item On utilise la fonction usuelle suivante : \pause $ (x^n)' = \textcolor{red}{n} \times \pause x^{\textcolor{red}{n}-1} $
				\pause
				\item Et la règle de dérivation suivante : \pause $ (\textcolor{red}{\lambda} u)' = \textcolor{red}{\lambda} u' $
				\pause
				\item \(f'(x) = \pause (\textcolor{green}{3}x^{\textcolor{red}{3}})' = \pause \textcolor{red}{3} \pause \times \pause \textcolor{green}{3} \pause x^{\textcolor{red}{3}-1} = \pause \textcolor{green}{9x^2}\)
				\pause
				\item \(f'(2) = 9 \times \pause 2^2 = \pause 36 \) 
			\end{enumerate}
			\pause
			Nous pouvons déterminer l'équation de la tangente : y = f'(a)(x-a) + f(a) \\
			\(y = \pause f'(2)(x-2) + f(2) = \pause 36(x-2) + 24 = \pause 36x - 72 + 24 \)
			\pause
			\[\textcolor{red}{y = 36x + 48}  \]
		\end{block}
	\end{frame}

	\begin{frame}
		\frametitle{b) $g(x) = \sqrt{x+1}$ au point d'abscisse 5}
		\pause
		Vous ne pouvez pas encore faire cet exemple cette année. \pause
		Vous le verrez l'année prochaine :)
	\end{frame}

	\begin{frame}
		\frametitle{c) $i(x) = \dfrac{1}{x}$ au point d'abscisse 3}
		\begin{block}{1) Déterminer la fonction dérivée f'(x) de f(x)}
			\pause
			\begin{enumerate}[]
				\pause
				\item f(x) est une fonction définie sur $\mathbb{R}$ du type $\dfrac{u}{v}$ tel que $u(x) = 1$ et $v(x)=x$
				\pause
				\item On va utiliser la règle suivante : $\left( \dfrac{u}{v} \right)' = \dfrac{u'v-v'u}{v^2} $
				\pause
				\item On utilise la fonction usuelle suivante : \pause $ (x^n)' = \textcolor{red}{n} \times \pause x^{\textcolor{red}{n}-1} $
				\item Notre but est de déterminer u' et v'
				\pause
				\item \(u'(x) = 0\)
				\pause
				\item \(v'(x) = (x^1)' = 1 \pause \times \pause x^{1-1} = \pause x^0 = 1\)
				\pause 
				\item \(i'(x) = \pause \dfrac{u'v-v'u}{v^2} = \pause \dfrac{0 \times x- 1 \times 1}{x^2} = \pause \textcolor{green}{\dfrac{-1}{x^2}} \)
			\end{enumerate}
		\end{block}
	\end{frame}

	\begin{frame}
		\frametitle{d) $j(x) = x^2 - 5x +7$ au point d'abscisse -2}
		\begin{block}{1) Déterminer la fonction dérivée j'(x) de j(x)}
			\pause
			j(x) est une fonction polynôme définie sur $\mathbb{R}$, \pause donc elle est dérivable sur $\mathbb{R}$
			\begin{enumerate}[]
				\pause
				\item On utilise la fonction usuelle suivante : \pause $ (x^n)' = \textcolor{red}{n} \times \pause x^{\textcolor{red}{n}-1} $
				\pause
				\item Et la règle de dérivation suivante : \pause $ (\textcolor{red}{\lambda} u)' = \textcolor{red}{\lambda} u' $
				\pause
				\item \(j'(x) = \pause 2 \times \pause x^{ \pause 2 \pause -1} \pause - 5 \pause \times \pause 1 \pause \times \pause x^{\pause 1 \pause -1} \pause + 0 \)
				\pause
				\item \(j'(x) = \pause \textcolor{green}{2x - 5} \)
			\end{enumerate}
		\end{block}
	\end{frame}
\end{document}

