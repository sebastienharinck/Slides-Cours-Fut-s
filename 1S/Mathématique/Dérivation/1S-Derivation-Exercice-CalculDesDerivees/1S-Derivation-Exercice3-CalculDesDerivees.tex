% ==================================================
% 1S - DERIVATION - EXERCICE 3 - Calcul Des Derivees
% ==================================================

% ADD BEAMER THEME
\documentclass[t]{beamer}

% ADD PACKAGES
\usepackage{etex}
\usepackage[english]{babel}
\usepackage[utf8]{inputenc}
\usepackage{pstricks-add}
\usepackage{pst-eucl}

\beamertemplatenavigationsymbolsempty

% DEFINE COLORS
\definecolor{red}{HTML}{DD0000}
\definecolor{green}{RGB}{51,110,23}
\setbeamercolor{block title example}{fg=green}
	
% DOCUMENT
\begin{document}

	\begin{frame}
		\frametitle{1ère S - Dérivation - Exercice 3 : Calcul des Dérivées}
		Dériver les 5 fonctions suivantes :
		\begin{enumerate}
			\item f définie sur $\mathbb{R}$ par $f(x) = 3x^4 - 5x^2 + 3x + 4 $
			\item h définie sur $[0;+\infty[$ par $h(x) = (x^2+1)\sqrt{x} $
			\item i définie sur $\mathbb{R} - \left\lbrace  \dfrac{1}{3} \right\rbrace $ par $i(x) = \dfrac{7x+7}{3x-1} $
			\item j définie sur $\mathbb{R}$ par $j(x) = (4x^2+3x+1)^4$
			\item k définie sur $\mathbb{R}$ par $k(x) = (x+4)^2 (x^2+3x+2) $
		\end{enumerate}
	\end{frame}

	\begin{frame}
		\frametitle{$f(x) = 3x^4 - 5x^2 + 3x + 4 $ sur $\mathbb{R}$}
		\pause
		Pour dériver une fonction, vous devez le faire en 2 étapes : 
		\pause
		\begin{enumerate}
			\item Justifier la \textcolor{red}{dérivabilité de la fonction} \pause sur un \textcolor{red}{intervalle de dérivabilité}
			\pause
			\item Calculer la dérivée en utilisant les \textcolor{red}{fonctions usuelles} \pause et les \textcolor{red}{règles de dérivation}
			\pause
		\end{enumerate}

		\begin{block}{1) Justifier la dérivabilité de la fonction sur un intervalle de dérivabilité :}
			\pause
			$f(x) = 3x^4 - 5x^2 + 3x + 4$ est une fonction polynôme. \pause Elle est donc dérivable sur $\mathbb{R}$
			\pause
		\end{block}

			\begin{block}{2) Calculer la dérivée en utilisant les fonctions usuelles et les règles de dérivation}
			\pause
			\begin{enumerate}[]
				\item On utilise la dérivée de la fonction usuelle suivante :
				\pause
				\item \(\textcolor{green}{ (x^n)' = n \times x^{n-1} }\)
				\pause
				\item On utilise la règle de dérivation suivante :
				\pause
				\item \(\textcolor{green}{(u+v)' = u' + v'}$ \pause et $\textcolor{green}{ (\lambda u)' = \lambda u'}\)
				\pause
			\end{enumerate}
		\end{block}
	\end{frame}

	\begin{frame}
		\frametitle{$f(x) = 3x^4 - 5x^2 + 3x + 4 $ sur $\mathbb{R}$}
		\pause
		\begin{enumerate}[]
			\item \(f'(x) = 4 \times 3x^{4-1} + 2 \times (-5)x^{2-1} + 3 + 0 \)
			\pause
			\item \(\textcolor{red}{f'(x) = 12x^{3} -10x + 3} \)
			\pause
		\end{enumerate}
	\end{frame}

	\begin{frame}
		\frametitle{$h(x) = (x^2+1)\sqrt{x} $ sur $[0;+\infty[$}
			\begin{block}{1) Justifier la dérivabilité de la fonction sur un intervalle de dérivabilité :}
			\pause
			h(x) est une fonction de la forme $u \times v$ où $u(x)= (x^2+1) $ et $v(x)= \sqrt{x}$. u(x) est une fonction polynôme définie et dérivable $[0;+\infty[$. x(x) est la fonction racine carrée, définie sur $[0;+\infty[$ et dérivable sur $\textcolor{red}{]0;+\infty[}$. On en déduit que h(x) sera dérivable sur $]0;+\infty[$.
			\pause
		\end{block}

		\begin{block}{2) Calculer la dérivée en utilisant les fonctions usuelles et les règles de dérivation}
			\pause
			\begin{enumerate}[]
				\item On utilise la dérivée de la fonction usuelle suivante :
				\pause
				\item \(\textcolor{green}{ (x^n)' = n \times x^{n-1} }\)
				\pause
				\item \((\sqrt{x})' = \dfrac{1}{2\sqrt{x}} \)
				\pause
				\item On utilise la règle de dérivation suivante :
				\pause
				\item \(\textcolor{green}{(uv)' = u'v + v'u}$ \pause et $\textcolor{green}{ (\lambda u)' = \lambda u'}\)
				\pause
			\end{enumerate}
		\end{block}
	\end{frame}

	\begin{frame}
		\frametitle{$h(x) = (x^2+1)\sqrt{x} $ sur $[0;+\infty[$}
		\pause
		\begin{enumerate}[]
			\item $\textcolor{green}{u(x) = (x^2+1)}$ et $\textcolor{green}{v(x) = \sqrt{x}} $ 
			\pause
			\item $\textcolor{green}{u'(x) = 2x}$ et $\textcolor{green}{v'(x) = \dfrac{1}{2\sqrt{x}}} $ 
			\pause
			\item \(h'(x) = u'v + v'u \)
			\pause
			\item \(h'(x) = 2x \times \sqrt{x} +  \dfrac{1}{2\sqrt{x}} \times (x^2+1) \)
			\pause
			\item \(h'(x) = 2x\sqrt{x} +  \dfrac{x^2+1}{2\sqrt{x}}\)
			\pause
			\item \(h'(x) = \dfrac{2x\sqrt{x}\textcolor{red}{(2\sqrt{x})}}{\textcolor{red}{2\sqrt{x}}} +  \dfrac{x^2+1}{2\sqrt{x}}\)
			\pause
			\item \(h'(x) = \dfrac{4x^2}{2\sqrt{x}} +  \dfrac{x^2+1}{2\sqrt{x}}\)
			\pause
			\item \(h'(x) = \dfrac{4x^2 + x^2+1}{2\sqrt{x}}\)
			\pause
			\item \[\textcolor{red}{h'(x) = \dfrac{5x^2+1}{2\sqrt{x}}}\]
		\end{enumerate}
	\end{frame}

	\begin{frame}
		\frametitle{$i(x) = \dfrac{7x+7}{3x-1} $ sur $\mathbb{R} - \left\lbrace  \dfrac{1}{3} \right\rbrace $}
	\end{frame}

	\begin{frame}
		\frametitle{$j(x) = (4x^2+3x+1)^4$ sur $\mathbb{R}$}
	\end{frame}

	\begin{frame}
		\frametitle{ $k(x) = (x+4)^2 (x^2+3x+2) $ sur $\mathbb{R}$}
	\end{frame}
\end{document}

