% ==========================================
% 1S - DERIVATION - EXERCICE - NOMBRE DERIVE
% ==========================================

% ADD BEAMER THEME
\documentclass[t]{beamer}

% ADD PACKAGES
\usepackage{etex}
\usepackage[english]{babel}
\usepackage[utf8]{inputenc}
\usepackage{pstricks-add}
\usepackage{pst-eucl}

\beamertemplatenavigationsymbolsempty

% DEFINE COLORS
\definecolor{red}{HTML}{DD0000}
\definecolor{green}{RGB}{51,110,23}
\setbeamercolor{block title example}{fg=green}

% DOCUMENT
\begin{document}

	\begin{frame}[label=pagebanale]
		\frametitle{Dérivation - Exercice 1 : Nombre Dérivé}
		\pause
		Déterminer les équations des tangentes aux courbes des fonctions suivantes :
		\pause
		\begin{enumerate}[a)]
			\item<+-> $f(x) = 3x^3$ au point d'abscisse 2
			\item<+-> $g(x) = \sqrt{x+1}$ au point d'abscisse 5
			\item<+-> $i(x) = \dfrac{1}{x}$ au point d'abscisse 3
			\item<+-> $j(x) = x^2 - 5x +7$ au point d'abscisse -2
		\end{enumerate}
	\end{frame}

	\begin{frame}
		\frametitle{a) $f(x) = 3x^3$ au point d'abscisse 2}
		\pause
		\begin{block}{}
			Pour déterminer l'équation de la tangente d'une fonction, vous devez (\underline{pour le moment}) :
			\pause
			\begin{enumerate}[1.]
				\item Calculer le taux d'accroissement de la fonction entre a et a+h
				\pause
				\item Calculer le nombre dérivé, \pause en calculant la limite du \textcolor{green}{taux d'accroissement} lorsque $h \rightarrow 0$
				\pause
				\item Puis finalement, déterminer l'équation de la tangente
			\end{enumerate}
		\end{block}
		\pause
		\begin{block}{1) Calculer le taux d'accroissement :}
			\pause
			\begin{enumerate}[]
				\item<+-> $\cfrac{f(a+h)-f(a)}{h}$ où a = 2.
				\item<+-> \(\cfrac{f(2+h)-f(2)}{h} \)
			\end{enumerate}
		\end{block}
	\end{frame}

	\begin{frame}
		\frametitle{a) $f(x) = 3x^3$ au point d'abscisse 2}
		\pause
		\begin{enumerate}[]
			\item \(\textcolor{red}{f(2) =  3 \times 2^3 = 24}\)
			\pause
			\item \(f(2+h) =  3 \times (2+h)^3 \pause = 3 \times (2+h)^2(2+h) \)
			\pause
			\item \(f(2+h) = 3(2^2 + 2 \times 2 \times h + h^2)(2+h)\)
			\pause
			\item \(f(2+h) = 3(4 + 4h + h^2)(2+h)\)
			\pause
			\item \(f(2+h) = 3(8 + 4h + 8h + 4h^2 + 2h^2 + h^3)\)
			\pause
			\item \(f(2+h) = 3(h^3 + 6h^2 + 12h + 8)\)
			\pause
			\item \(\textcolor{red}{f(2+h) = 3h^3 + 18h^2 + 36h + 24}\)
			\pause
		\end{enumerate}
		\begin{block}{Ce qui nous donne :}
			\pause
			\begin{enumerate}[]
				\item \(f(2+h) - f(2) = \pause 3h^3 + 18h^2 + 36h + 24 - 24 \)
				\pause
				\item \(f(2+h) - f(2) = \pause 3h^3 + 18h^2 + 36h \)
			\end{enumerate}
		\end{block}
		\pause
		\( \dfrac{f(2+h) - f(2)}{h} = \pause \dfrac{3h^3 + 18h^2 + 36h}{h} = \pause \dfrac{h(3h^2 + 18h + 36)}{h}\)
		\pause
		\[=\textcolor{red}{3h^2 + 18h + 36}\]
	\end{frame}

	\begin{frame}
		\frametitle{a) $f(x) = 3x^3$ au point d'abscisse 2}
		\pause
		Le taux d'accroissement de la fonction f entre 2 et 2+h est égal à \pause $\textcolor{red}{3h^2 + 18h + 36}$.\pause On peut facilement en déduire le nombre dérivé de f \pause en 2 : f'(2).
		\pause
		Rappel de la formule de la dérivée d'une fonction en a : 
		\pause
		\begin{enumerate}[]
			\item<+-> \(\textcolor{red}{f'(a) = \lim\limits_{h \rightarrow 0}\dfrac{f(a+h)-f(a)}{h}}\)
			\item<+-> \(f'(2) = \lim\limits_{h \rightarrow 0} (3h^2 + 18h + 36) \pause = \textcolor{green}{36} \)
			\pause
		\end{enumerate}

		Equation de la tangente T en a :
		\pause
		\begin{enumerate}[]
			\item<+-> \[\textcolor{red}{y = f'(a)(x-a) + f(a)}\]
			\item<+-> \(y=f'(2)(x-2) + f(2) = 36(x-2) + 24 = 36x -48 \)
			\item<+-> \[\textcolor{green}{y = 36x -48} \]
		\end{enumerate}
	\end{frame}

	\begin{frame}
		\frametitle{b) $g(x) = \sqrt{x+1}$ au point d'abscisse 5}
		\pause
		Calculer le taux d'accroissement de la fonction entre a et a+h. \pause Où a=5.
		\pause
		On cherche à calculer \( \dfrac{g(5+h) - g(5)}{h} \)
		\pause
		\begin{enumerate}[]
			\item<+-> \(g(5) = \sqrt{5+1} = \sqrt{6} \)
			\item<+-> \(g(5+h) = \sqrt{5+h+1} = \sqrt{6+h} \)
			\item<+-> \(g(5+h) - g(5) = \sqrt{6+h} - \sqrt{6} \)
			\item<+-> \( \dfrac{(5+h) - g(5)}{h} = \dfrac{\sqrt{6+h} - \sqrt{6}}{h} \)
			\item<+-> L'astuce ici consiste à multiplier le numérateur et le dénominateur par \color{red}l'expression conjuguée
			\item<+-> L'expression conjuguée de $\textcolor{red}{a+b}$ est $\textcolor{red}{a-b}$
			\item<+-> Donc l'expression conjuguée de $\sqrt{6+h} - \sqrt{6}$ est donc $\sqrt{6+h} + \sqrt{6}$
			\item<+-> \(\dfrac{(\sqrt{6+h} - \sqrt{6}) \textcolor{green}{\times (\sqrt{6+h} + \sqrt{6})}}{h \textcolor{green}{(\sqrt{6+h} + \sqrt{6})}}\)
		\end{enumerate}
	\end{frame}

	\begin{frame}[label=pagebanale]
		\frametitle{b) $g(x) = \sqrt{x+1}$ au point d'abscisse 5}
		\pause
		\begin{enumerate}[]
			\item<+-> \[\dfrac{(\sqrt{6+h} - \sqrt{6}) \textcolor{green}{\times (\sqrt{6+h} + \sqrt{6})}}{h \textcolor{green}{(\sqrt{6+h} + \sqrt{6})}}\]
			\item<+-> \[\dfrac{(6+h) + \sqrt{6+h}\sqrt{6} - \sqrt{6}\sqrt{6+h} - 6}{h (\sqrt{6+h} + \sqrt{6})} \]
			\item<+-> \[\dfrac{6+h - 6}{h (\sqrt{6+h} + \sqrt{6})} \]
			\item<+-> \[\dfrac{h}{h (\sqrt{6+h} + \sqrt{6})} \]
			\item<+-> \[\dfrac{1}{\sqrt{6+h} + \sqrt{6}} \]
		\end{enumerate}
	\end{frame}

	\begin{frame}[label=pagebanale]
		\frametitle{b) $g(x) = \sqrt{x+1}$ au point d'abscisse 5}
		\pause
		Le taux d'accroissement de la fonction g entre 5 et 5+h est égal à \pause $\textcolor{green}{\dfrac{1}{\sqrt{6+h} + \sqrt{6}}}$.\pause On peut facilement en déduire le nombre dérivé de g en 5 : g'(5) \\
		\pause
		Rappel de la formule de la dérivée d'une fonction en a :
		\pause 
		\begin{enumerate}[]
			\item<+-> \(\textcolor{red}{g'(a) = \lim\limits_{h \rightarrow 0}\dfrac{g(a+h)-g(a)}{h}}\)
			\item<+-> \(g'(5) = \lim\limits_{h \rightarrow 0}\dfrac{1}{\sqrt{6+h} + \sqrt{6}} = \pause \dfrac{1}{\sqrt{6} + \sqrt{6}} = \pause \textcolor{red}{\dfrac{1}{2\sqrt{6}}}\)
			\pause
		\end{enumerate}

		Equation de la tangente T en a :
		\pause
		\begin{enumerate}[]
			\item<+-> \[\textcolor{red}{y = f'(a)(x-a) + f(a)}\]
			\item<+-> \(y=g'(5)(x-5) + g(5) = \pause \dfrac{1}{2 \sqrt{6}}(x-5) + \sqrt{6}  = \pause \dfrac{1}{2 \sqrt{6}}x + \dfrac{-5}{2 \sqrt{6}} + \sqrt{6} \)
			\pause
			\item<+-> \[\textcolor{green}{y = \dfrac{1}{2\sqrt{6}}x + \dfrac{7}{\sqrt{6}}}\]
		\end{enumerate}
	\end{frame}

	\begin{frame}
		\frametitle{c) $i(x) = \dfrac{1}{x}$ au point d'abscisse 3}
		\pause
		Calculer le taux d'accroissement de la fonction entre a et a+h. \pause Où a=3.
		\pause
		On cherche à calculer \( \dfrac{i(3+h) - i(3)}{h} \)
		\pause
		\begin{enumerate}[]
			\item<+-> \(i(3) = \dfrac{1}{3}. \pause i(3+h) = \dfrac{1}{3+h}\) 
			\pause
			\item<+-> \[\dfrac{i(3+h) - i(3)}{h} = \pause \dfrac{\dfrac{1}{3+h} - \dfrac{1}{3} }{h} = \pause \dfrac{\dfrac{1}{3+h}-\dfrac{1\textcolor{red}{(1+\dfrac{h}{3})}}{3\textcolor{red}{(1+\dfrac{h}{3})}}}{h}\]
			\pause
			\item<+-> \[ = \dfrac{\dfrac{1}{3+h}-\dfrac{1+\dfrac{h}{3}}{3+h}}{h} = \pause \dfrac{\dfrac{1-1-\dfrac{h}{3}}{3+h}}{h} \]
		\end{enumerate}
	\end{frame}

	\begin{frame}
		\frametitle{c) $i(x) = \dfrac{1}{x}$ au point d'abscisse 3}
		\pause
		\begin{enumerate}[]
			\item<+-> \[= \dfrac{\dfrac{1-1-\dfrac{h}{3}}{3+h}}{h} \]
			\item<+-> \[= \dfrac{\dfrac{-(\dfrac{h}{3})}{3+h}}{\dfrac{h}{1}} \]
			\item<+-> \[= \dfrac{(\dfrac{-1}{3})h}{3+h} \times \pause \textcolor{green}{ \dfrac{1}{h}}\]
			\pause
			\item<+-> \[= \dfrac{\dfrac{-1}{3}}{3+h}\]
		\end{enumerate}
	\end{frame}

	\begin{frame}
		\frametitle{c) $i(x) = \dfrac{1}{x}$ au point d'abscisse 3}
		\pause
		Le taux d'accroissement de la fonction i entre 3 et 3+h est égal à \pause $\textcolor{green}{\dfrac{\dfrac{-1}{3}}{3+h}}$ \pause. On peut facilement en déduire la valeur de i'(3) \\
		\pause
		\begin{enumerate}[]
			\item<+-> \(i'(3) = \pause \lim\limits_{h \rightarrow 0}\dfrac{\dfrac{-1}{3}}{3+h} = \pause \dfrac{\dfrac{-1}{3}}{3} = \pause \dfrac{\dfrac{-1}{3}}{\dfrac{3}{1}} = \pause \dfrac{-1}{3} \times \dfrac{1}{3} = \pause \textcolor{red}{\dfrac{-1}{9}}\)
			\pause
		\end{enumerate}
		Equation de la tangente T en a :
		\pause
		\begin{enumerate}[]
			\item<+-> \[\textcolor{red}{y = f'(a)(x-a) + f(a)}\]
			\item<+-> \(y=i'(3)(x-3) + i(3) = \pause \dfrac{-1}{9}(x-3) + \dfrac{1}{3} = \pause \dfrac{-1}{9}x + \dfrac{1}{3} + \dfrac{1}{3} = \pause \dfrac{-1}{9}x + \dfrac{2}{3} \)
			\pause
			\item<+-> \[\textcolor{green}{y = \dfrac{-1}{9}x + \dfrac{2}{3}}\]
		\end{enumerate}
	\end{frame}

	\begin{frame}
		\frametitle{d) $j(x) = x^2 - 5x +7$ au point d'abscisse -2}
		\pause
		Calculer le taux d'accroissement de la fonction entre a et a+h. \pause Où a=-2.
		\pause
		On cherche à calculer \( \dfrac{j(3+h) - j(3)}{h} \)
		\pause
		\begin{enumerate}[]
			\item<+-> \(j(-2)=(-2)^2 - 5\times (-2) + 7 = 4 + 10 + 7 = \textcolor{red}{20} \)
			\item<+-> \(j(-2+h)=(-2+h)^2 - 5(-2+h) + 7 \)
			\item<+-> \(j(-2+h)=((-2)^2+ 2 \times (-2) \times h + h^2) + 10 - 5h + 7 \)
			\item<+-> \(j(-2+h)=4 - 4h + h^2 + 10 - 5h + 7 \)
			\item<+-> \(\textcolor{red}{j(-2+h)= h^2 -9h + 20} \)
			\item<+-> \[\dfrac{f(-2+h)-f(-2)}{h}\]
			\item<+-> \[=\dfrac{h^2-9h+20-20}{h}\]
			\item<+-> \[=\dfrac{h^2-9h}{h} = \pause \dfrac{h(h-9)}{h} = \pause \textcolor{green}{h-9} \]
		\end{enumerate}
	\end{frame}

	\begin{frame}
		\frametitle{d) $j(x) = x^2 - 5x +7$ au point d'abscisse -2}
		\pause
		Le taux d'accroissement de la fonction j entre -2 et -2+h est égal à \pause $\textcolor{green}{h-9}.$ \pause On peut facilement en déduire la valeur de j'(-2) \\
		\pause
		\begin{enumerate}[]
			\item<+-> \(j'(-2) = \lim\limits_{h \rightarrow 0}h-9 = \pause -9 \)
		\end{enumerate}
		\pause
		Equation de la tangente T en a :
		\pause
		\begin{enumerate}[]
			\item<+-> \[\textcolor{red}{y = f'(a)(x-a) + f(a)}\]
			\item<+-> \(y=j'(-2)(x-(-2)) + j(-2) \)
			\item<+-> \(y= -9(x+4) + 20\)
			\item<+-> \(y= -9x-36+20\)
			\item<+-> \(y= -9x-16 \)
			\pause
			\item<+-> \[\textcolor{green}{y = -9x-16 }\]
		\end{enumerate}
	\end{frame}

\end{document}