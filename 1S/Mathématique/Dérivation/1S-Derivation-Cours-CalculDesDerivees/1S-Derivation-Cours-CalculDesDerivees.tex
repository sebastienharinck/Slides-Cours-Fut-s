% ==============================================
% 1S - DERIVATION - COURS - CALCUL DES DERIVEES
% ==============================================

% ADD BEAMER THEME
\documentclass[t]{beamer}

% ADD PACKAGES
\usepackage{etex}
\usepackage[english]{babel}
\usepackage[utf8]{inputenc}
\usepackage{pstricks-add}
\usepackage{pst-eucl}

\beamertemplatenavigationsymbolsempty

% DEFINE COLORS
\definecolor{red}{HTML}{DD0000}
\definecolor{green}{RGB}{51,110,23}
\setbeamercolor{block title example}{fg=green}

% DOCUMENT
\begin{document}

	\begin{frame}[label=pagebanale]
		\frametitle{Fonction Dérivée :}
		\pause
		\begin{enumerate}[a)]
			\item<+-> Définition de la dérivation d'une fonction
			\item<+-> Les 3 fonctions usuelles
			\item<+-> Les 5 règles de dérivation
		\end{enumerate}
	\end{frame}

	\begin{frame}[label=pagebanale]
		\frametitle{a) Dériver une fonction : }
		\pause
		\begin{block}{Définition :}
		\pause
		Soit f une fonction définie sur un intervalle I. \pause Si f est dérivable $\forall x \in I$, \pause on dit que f est dérivable sur I. \pause La fonction dérivée de f est la fonction f' qui à tout x de I \pause associe le nombre de f'(x).
		\pause
		\end{block}
	\end{frame}

	\begin{frame}[label=pagebanale]
		\frametitle{b) Les 3 fonctions usuelles}
		\pause
		\begin{tabular}{|c|c|c|}
			\hline
				\textbf{Intervalle de dérivation} & \textbf{f(x)} & \textbf{f'(x)} \\
			\hline
				$\mathbb{R}$ & k(constante réelle) & 0 \\
			\hline
				$\mathbb{R}$ & $x^{\textcolor{red}{n}}$ & $\textcolor{red}{n} \times x^{(\textcolor{red}{n}-1)} $ \\
			\hline
				$\left] 0;+ \infty \right[ $ & $\sqrt{x}$ & $\frac{1}{2 \sqrt{x}}$ \\
			\hline 
		\end{tabular}
		\pause
		\begin{block}{Exemples : }
			\pause
			\begin{enumerate}[a)]
				\item $ f(x)=4 \Rightarrow \pause f'(x)= 0 $
				\pause
				\item $ f(x)=493.9 \Rightarrow \pause f'(x)= 0 $
				\pause
				\item $ f(x)=x = \pause x^1 \Rightarrow \pause f'(x)= \textcolor{green}{1} \pause \times \pause x^{(\pause \textcolor{green}{1} \pause-1)} = \pause 1 \times x^0 = \pause 1 \times 1 = 1 $
				\pause
				\item $ f(x)=x^2 \Rightarrow \pause f'(x)= \pause \textcolor{green}{2} \pause \times \pause x^{(\textcolor{green}{2} \pause -1)} = \pause 2 \times x^1 = \pause 2x $
				\pause
				\item $ f(x)=x^7 \Rightarrow \pause f'(x)= \pause \textcolor{green}{7} \pause \times \pause x^{(\textcolor{green}{7}-1)} = \pause 7 \times x^6 = \pause 7x^6 $
				\pause
				\item $ f(x)=\sqrt{x} \Rightarrow \pause f'(x)= \frac{1}{2 \sqrt{x}}  $
				\pause
			\end{enumerate}
		\end{block}
	\end{frame}

	\begin{frame}[label=pagebanale]
		\frametitle{c) Les 5 règles de dérivation}
		\pause
		Soit u et v deux fonctions.
		\begin{tabular}{|c|c|c|}
			\hline
				\textbf{forme de f(x)} & \textbf{f'(x)} \\
			\hline
				$u+v$ & $u' + v'$ \\
			\hline
				$\lambda u$ & $\lambda u'$ \\
			\hline
				$uv $ & $u'v +v'u$ \\
			\hline 
				$\dfrac{u}{v}$ & $\dfrac{u'v-v'u}{v^2}$ \\
			\hline
				$u^n$ & $n \times u' u^{(n-1)}$ \\
			\hline
		\end{tabular} 
		\pause
		\begin{block}{Exemples :}
			\begin{enumerate}
				\pause
				\item $f(x) = x^2 + x$. \pause Dans notre cas, on remarque que f(x) est de la forme u + v. \pause Comme $(x^2)' = 2x$ \pause et $(x)' = 1$. \pause On en déduit que $\textcolor{green}{f'(x) = 2x + 1}$. \pause
				\item $g(x) = 42 \sqrt{x}$. \pause Dans notre cas, on remarque que g(x) est de la forme $\lambda u$ \pause où $\lambda = 42$ \pause et $u = \sqrt{x}$. \pause Comme $u' = (\sqrt{x})' = \pause \frac{1}{2\sqrt{x}}$, \pause on en déduit que $g'(x) = \pause 42 \pause \times \pause \dfrac{1}{2\sqrt{x}} \pause = \dfrac{42}{2\sqrt{x}} = \textcolor{green}{\dfrac{24}{\sqrt{x}}} \pause $
			\end{enumerate}
		\end{block}
	\end{frame}

	\begin{frame}
		\frametitle{Exemple avec (uv)' = u'v + v'u}
		\pause
		\begin{block}{Dériver h(x)}
			$h(x) = x^3 \sqrt{x}$. \pause Il s'agit d'une fonction de la forme uv \pause où u et v sont deux fonctions \pause telles que $u(x) = \textcolor{green}{x^3}$ \pause et $v(x) = \textcolor{green}{\sqrt{x}}$. \pause Nous allons utiliser la formule (uv)' = u'v + v'u. \pause Calculons :
			\begin{enumerate}{}
				\item<+-> \(u'(x)= 3 \times x^{(3-1)} = \textcolor{green}{3x^2} \)
				\item<+-> \(v'(x) = \textcolor{green}{\frac{1}{2 \sqrt{x}}} \) 
			\end{enumerate}
			\pause
			A partir d'ici, il suffit de remplacer : \pause
			\begin{enumerate}[]
				\pause
				\item\(g'(x) = u'v + v'u = \pause 3x^2 \pause \times \pause \sqrt{x} \pause + \pause \frac{1}{2 \sqrt{x}} \pause \times \pause x^3\)
				\pause
				\item \(g'(x) = 3x^2 \sqrt{x} + \frac{x^3}{2 \sqrt{x}} \)
				\pause
				\item\(g'(x) = \dfrac{3x^2 \sqrt{x} \pause \times \textcolor{red}{2\sqrt{x}}\pause}{\textcolor{red}{2\sqrt{x}}} \pause + \frac{x^3}{2 \sqrt{x}} \)
				\pause
				\item \(g'(x) = \dfrac{6x^3 + x^3}{2\sqrt{x}} = \pause \textcolor{green}{\dfrac{7x^3}{2\sqrt{x}}} \)
			\end{enumerate}	
		\end{block}
	\end{frame}

	\begin{frame}
		\frametitle{Exemple avec $\frac{u}{v}$ }
		\pause
		\begin{block}{Dériver i(x)}
			$i(x) = \frac{x^2 + x}{3x}$. \pause Il s'agit d'une fonction de la forme $\frac{u}{v}$ \pause où u et v sont deux fonctions \pause telles que $u(x) = \textcolor{green}{x^2 + x}$ \pause et $v(x) = \textcolor{green}{3x}$. \pause Nous allons utiliser la formule $\textcolor{red}{\left( \frac{u}{v} \right)' = \frac{u'v - v'u}{v^2}}$. \pause Calculons :
			\pause
			\begin{enumerate}{}
				\item<+-> \(u'(x)= 2x + 1 \)
				\item<+-> \(v'(x) = 3 \) 
			\end{enumerate}
			\pause
			A partir d'ici, il suffit de remplacer : \pause
			\begin{enumerate}[]
				\item \(i'(x) = \pause \frac{u'v - v'u}{v^2} = \pause \frac{(2x+1) \pause \times \pause 3x \pause - \pause 3 \pause \times \pause (x^2+x) \pause}{(3x)^2}\)
				\pause
				\item \(i'(x) = \pause \frac{6x^2 \pause + 3x \pause - \pause 3x^2 \pause - 3x \pause}{(3x)^2} \)
				\pause
				\item \(i'(x) = \pause \frac{3x^2}{(3x)^2} \)
				\pause
				\item \(i'(x) = \pause \frac{3x^2}{9x^2} \)
				\pause
				\item \(i'(x) = \pause \textcolor{green}{\frac{1}{3}} \)
			\end{enumerate}	
		\end{block}
	\end{frame}
\end{document}