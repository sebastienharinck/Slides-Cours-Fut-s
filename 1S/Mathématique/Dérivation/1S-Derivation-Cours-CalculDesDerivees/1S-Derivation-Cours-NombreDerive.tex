\documentclass[t]{beamer}
\usepackage{etex}
\usepackage[english]{babel}
\usepackage[utf8]{inputenc}
\usepackage{pstricks-add}
\usepackage{pst-eucl}
\usetheme{Warsaw}

\definecolor{red}{HTML}{DD0000}
\definecolor{green}{RGB}{51,110,23}

\title{1ère S : Dérivation - Cours 1 : Nombre dérivé}
\author{www.cours-futes.com}
\institute{Sébastien Harinck}
\date{}

\begin{document}
\begin{frame}
\titlepage
\end{frame}

\begin{frame}[label=pagebanale]
\frametitle{Nombre Dérivé :}
\pause
\begin{enumerate}[a)]
\item<+-> Taux d'accroissement d'une fonction
\item<+-> Comment dériver une fonction en un point
\item<+-> Comment déterminer l'équation d'une tangente à une courbe
\end{enumerate}
\end{frame}

\begin{frame}[label=pagebanale]
\frametitle{a)  Taux d'accroissement d'une fonction : }
\pause
\begin{block}{Définition :}
\pause
Soit f une fonction définie sur un intervalle I, \pause a et b $ \in I$, \pause tel que $a \ne b$. \pause On pose $h = b - a \pause \Leftrightarrow \pause \textcolor{red}{ b = a + h}$. \pause
On dit que le \textcolor{red}{taux d'accroissement} de f \pause entre a et a + h \pause est le nombre : 
\pause
\[\frac{f(a+h) - f(a)}{h}\]
\pause
\end{block}
\begin{block}{Taux d'accroissement}
\pause
{\Huge \textcolor{green}{\[\frac{f(a+h) - f(a)}{h}\]}}
\end{block}
\end{frame}

\begin{frame}[label=pagebanale]
\frametitle{b) Dériver une fonction en un point}
\pause
\begin{block}{Définition :}
\pause
Le \textcolor{red}{nombre dérivé de f en a} est la limite, \pause si elle existe, \pause du \textcolor{green}{taux d'accroissement} \pause
\(\frac{f(a+h) - f(a)}{h}\)
\pause
lorsque h tend vers 0. \pause 
On le note f'(a) \pause  
et on dit alors que \textcolor{red}{f est dérivable en a}.
\pause
\end{block}
\begin{block}{On note :}
\pause
{\Huge \textcolor{green}{\[ f'(a) = \lim\limits_{h \rightarrow 0}\frac{f(a+h) - f(a)}{h}\]}}
\end{block}
\end{frame}

\begin{frame}[label=pagebanale]
\frametitle{b) Dériver une fonction en un point}
\pause
\begin{exampleblock}{Exemple :}
\pause
Soit f la fonction définie sur
\( \left[ - 1 ; 4 \right]\)
par
\( f(x) = x^2 - 4x + 2\).


\pause
Calculer le nombre dérivé de f en 1.
\pause
\end{exampleblock}
\begin{block}{Le but est de calculer :}
\pause
\[  a = 1. \pause \
f'(1) =\lim\limits_{h \rightarrow 0}\frac{f(1+h) - f(1)}{h}\]
\pause
\end{block}
\begin{block}{On calcule donc :}
\pause
\begin{enumerate}[]
\item<+-> $\color<15->{red}{f(1) = 1^2 - 4 + 2 = -1}$
\item<+-> $f(1+ h) = (1+h)^2 - 4(1+h) + 2$
\item<+-> $f(1+ h) = (1^2 + 2 \times 1 \times h + h^2) \pause - 4 - 4h + 2$
\item<+-> $f(1+ h) = 1 + 2h + h^2  - 4 - 4h + 2$
\item<+-> $\color<15->{red}{f(1+ h) = -1 - 2h + h^2} $
\end{enumerate}
\end{block}
\end{frame}

\begin{frame}[label=pagebanale]
\frametitle{b) Calculer un nombre dérivé d'une fonction en un point}
\pause
\begin{exampleblock}{Exemple :}
\pause
Soit f la fonction définie sur
\( \left[ - 1 ; 4 \right]\)
par
\( f(x) = x^2 - 4x + 2\).
\pause
Calculer le nombre dérivé de f en 1.
\pause
\end{exampleblock}
\begin{block}{Le but est de calculer :}
\pause
\begin{enumerate}[]
\item<+-> \textcolor{green}{\(f'(1) = \lim\limits_{h \rightarrow 0}\frac{f(1+h) - f(1)}{h}.\ \pause f(1) = -1. \ \pause f(1+ h) = -1 - 2h + h^2 \pause.\)}
\item<+-> $f(1+ h) - f(1)  = -1 - 2h + h^2 - (-1) $
\item<+-> $f(1+ h) - f(1)  = -2h + h^2 $
\item<+-> $f(1+ h) - f(1)  = h(h - 2) $
\end{enumerate}
\pause
Donc : 
\( \frac{f(1+ h) - f(1)}{h} = \pause \cfrac{ h(h-2)}{h} \pause = h - 2.\) 
\pause
Lorsque h tend vers 0, \pause l'expression h - 2 tend vers -2. \pause
Donc la fonction f est dérivable en \textcolor{red}{1} \pause et a pour nombre dérivé : \pause
\textcolor{red}{\[f'(1) = \pause -2.\]}
\end{block}
\end{frame}


\begin{frame}[label=pagebanale]
\frametitle{c) Déterminer l'équation d'une tangente à une courbe :}
\pause
\begin{block}{Tangente :}
Soit f une fonction définie sur un intervalle I.
\pause
Soit A(a,f(a))et M(a+h,f(a+h)),
\pause
$f'(a) = \lim\limits_{h \rightarrow 0}\dfrac{f(a+h) - f(a)}{h}$ \pause donne le coefficient directeur de la droite (AM).
\pause
\end{block}
\begin{columns}
\begin{column}{0.45\textwidth}
\begin{block}{Tangente à la courbe x=1}
\begin{enumerate}[]
\item<+-> \(f'(1) = \cfrac{f(1+h) - f(1)}{h} \)
\item<+-> \(f'(1) = \cfrac{h(2+h)}{h} = 2+h.\)
\item<+-> $2+h \rightarrow 2$ quand $h \rightarrow 0. $
\item<+-> \(y=f'(1)(x-1) + f(1).\)
\item<+-> \(y=2(x-1) + 1 = \textcolor{red}{2x - 1}.\)
\end{enumerate}
\pause
\end{block}
\end{column}
\begin{column}{0.55\textwidth}
\begin{block}{Tangente à une courbe :}
\begin{pspicture}(-3,-1)(2.5,2.5)
\psset{xunit=1 cm, algebraic=true}
\psaxes{->}(0,0)(-2.5,-1)(2.5,2)
\psplot[linecolor=red,linewidth=1.5pt]{-1.5}{1.5}{x^2}
\rput(1,2){\textcolor{red}{$y=x^2$}} % et une légende
\pause
\psplot[linecolor=blue,linewidth=1.5pt]{0}{2}{2*x-1}
\psdots(1,1)
\end{pspicture}
\end{block}
\end{column}
\end{columns}
\end{frame}
\end{document}