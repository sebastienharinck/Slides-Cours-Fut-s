\documentclass[t]{beamer}
%\usepackage[orientation=paysage,width=200,height=120,scale=2]{beamerposter}
\usepackage{etex}
\usepackage[english]{babel}
\usepackage[utf8]{inputenc}
\usepackage{pstricks-add}
\usepackage{pst-eucl}
\beamertemplatenavigationsymbolsempty
% \usetheme{Warsaw}

\definecolor{red}{HTML}{DD0000}
\definecolor{green}{RGB}{51,110,23}
\setbeamercolor{block title example}{fg=green}

\title{1ère S : Dérivation - Cours 1 : Nombre dérivé}
\author{Sébastien Harinck}
\institute{www.cours-futes.com}
\date{}

\begin{document}

\begin{frame}
\frametitle{Dérivation - Cours 1 : Nombre Dérivé}
\begin{enumerate}[a)]
\item Taux d'accroissement d'une fonction
\item Comment dériver une fonction en un point
\item Comment déterminer l'équation d'une tangente à une courbe
\end{enumerate}
\end{frame}

\begin{frame}[label=pagebanale]
\frametitle{a)  Taux d'accroissement d'une fonction : }
\pause
\begin{block}{Définition :}
\pause
Soit f une fonction définie sur un intervalle I, \pause a et b $ \in I$, \pause tel que $a \ne b$. \pause On pose $\pause \textcolor{red}{ b = a + h}$. \pause
On dit que le \textcolor{red}{taux d'accroissement} de f \pause entre a et a + h \pause est le nombre : 
\pause
\end{block}
\begin{block}
{\Huge \textcolor{green}{\[\frac{f(a+h) - f(a)}{h}\]}}
\end{block}
\end{frame}

\begin{frame}[label=pagebanale]
\frametitle{b) Dériver une fonction en un point}
\pause
\begin{block}{Définition :}
\pause
Le \textcolor{red}{nombre dérivé de f en a} est la limite, \pause si elle existe, \pause du \textcolor{green}{taux d'accroissement} \pause
\(\frac{f(a+h) - f(a)}{h}\)
\pause
lorsque h tend vers 0. \pause 
On dit alors que \textcolor{red}{f est dérivable en a}. \pause On le note f'(a)
\pause
\end{block}
\begin{block}{Mathématiquement :}
\pause
{\Huge \textcolor{green}{\[ f'(a) = \lim\limits_{h \rightarrow 0}\frac{f(a+h) - f(a)}{h}\]}}
\end{block}
\end{frame}

\begin{frame}[label=pagebanale]
\frametitle{b) Dériver une fonction en un point}
\pause
\begin{exampleblock}{Exemple :}
\pause
Soit f la fonction définie sur $\mathbb{R}$ par $f(x) = 2x^2 + 3x + 4.$

\pause
Calculer le nombre dérivé de f en -1.
\pause
\end{exampleblock}
\begin{block}{Le but est de calculer :}
\pause
\[  a = -1. \pause \
f'(-1) =\lim\limits_{h \rightarrow 0}\frac{f(-1+h) - f(-1)}{h}\]
\pause
\end{block}
\begin{block}{On calcule donc :}
\pause
\begin{enumerate}[]
\item \(\color<25->{red}{f(-1) = 2 \times (-1)^2 + 3 \times (-1) + 4 \pause = 3} \pause \)
\item \(f(-1+ h) = 2 \times \pause (-1+h)^2 \pause + 3 \times \pause (-1+h) + \pause 4 \pause \)
\item \(f(-1+ h) = 2 \times \pause ((-1)^2 + \pause 2 \times (-1) \times h + \pause  h^2) \pause - 3 + 3h \pause + 4 \pause \)
\item \( f(-1+ h) = 2 \times (1 - 2h + h^2) -3 +3h + 4 \pause \)
\item \(f(-1+h) = 2 - 4h + 2h^2 -3 + 3h + 4 \pause \)
\item \(\color<25->{red}{f(-1+ h) = 2h^2 - h + 3}\)
\end{enumerate}
\end{block}
\end{frame}

\begin{frame}[label=pagebanale]
\frametitle{b) Calculer un nombre dérivé d'une fonction en un point}
\pause
\begin{exampleblock}{Exemple :}
\pause
Soit f la fonction définie sur $\mathbb{R}$ par $f(x) = 2x^2 + 3x + 4.$

\pause
Calculer le nombre dérivé de f en -1.
\pause
\end{exampleblock}
\begin{block}{Le but est de calculer :}
\pause
\begin{enumerate}[]
\item \textcolor{green}{\(f'(-1) = \lim\limits_{h \rightarrow 0}\frac{f(-1+h) - f(-1)}{h}  |  \ \pause f(-1) = 3  |   \ \pause f(-1+ h) = 2h^2 - h + 3 \pause\)}
\item \(f(-1+ h) - f(-1) \pause = 2h^2 -h + 3 - 3 \pause \)
\item \(f(-1+ h) - f(-1)  = 2h^2 -h \pause \)
\end{enumerate}
\pause
Donc : 
\( \frac{f(-1+ h) - f(-1)}{h} = \pause \cfrac{ 2h^2-h}{h} = \pause \cfrac{ h(2h-1)}{h} \pause = 2h - 1.\) 
\pause
Lorsque h tend vers 0, \pause l'expression 2h - 1 tend vers -1. \pause
Donc la fonction f est dérivable en \textcolor{red}{-1} \pause et a pour nombre dérivé : \pause
\textcolor{red}{\[f'(-1) = \pause -1.\]}
\end{block}
\end{frame}


\begin{frame}[label=pagebanale]
\frametitle{c) Déterminer l'équation d'une tangente à une courbe :}
\pause
\begin{block}{Tangente :}
Soit f une fonction définie sur un intervalle I.
\pause
$f'(a) = \lim\limits_{h \rightarrow 0}\dfrac{f(a+h) - f(a)}{h}$ \pause donne le coefficient directeur de la tangente à la courbe au point A.
\pause
\end{block}
\begin{block}{}
L'équation de la tangente à la courbe de la fonction f au point d'abscisse a est : \\
\pause
{\Huge \textcolor{green}{\[ y = f'(a)(x-a) + f(a)\]}}
\end{block}
\end{frame}

\begin{frame}
\frametitle{c) Déterminer l'équation d'une tangente à une courbe :}
\pause
\begin{block}{Exemple :}
\pause
Soit f une fonction définie sur $\mathbb{R}$ tel que $f(x) = x^2$.\pause Et C la courbe représentative de f. \\
\pause
Déterminer l'équation de la tangente à la courbe C au point d'abscisse 1.
\pause
\end{block}

\begin{columns}
\begin{column}{0.45\textwidth}
\begin{block}{Tangente à la courbe x=1}
\pause
\begin{enumerate}[]
\item<+-> \(f'(1) = \lim\limits_{h \rightarrow 0} \cfrac{f(1+h) - f(1)}{h} \)
\item<+-> \(f'(1) = \lim\limits_{h \rightarrow 0} \cfrac{h(2+h)}{h} \)
\item<+-> \(f'(1) = \lim\limits_{h \rightarrow 0} 2+h = \pause 2\)
\pause
\item<+-> \(\textcolor{green}{y=f'(1)(x-1) + f(1).}\)
\item<+-> \(y= \pause 2(x-1) + 1 \pause = \textcolor{red}{2x - 1}.\)
\end{enumerate}
\pause
\end{block}
\end{column}
\begin{column}{0.55\textwidth}
\begin{block}{Graphiquement :}
\begin{pspicture}(-3,-1)(2.5,3)
\psset{xunit=1 cm, algebraic=true}
\psaxes{->}(0,0)(-2.5,-1)(2.5,2)
\psplot[linecolor=red,linewidth=1.5pt]{-1.5}{1.5}{x^2}
\rput(-0.7,2){\textcolor{red}{$y=x^2$}} % et une légende
\pause
\psdots(1,1)
\pause
\psplot[linecolor=blue,linewidth=1.5pt]{0}{2}{2*x-1}
\pause
\rput(2.5,1.5){\textcolor{blue}{$y=2x-1$}} % et une légende
\pause
\psdots(-1,1)
\pause
\psdots(0,0)
\end{pspicture}
\end{block}
\end{column}
\end{columns}
	\end{frame}

\begin{frame}{L'essentiel :}
\pause
\begin{exampleblock}{Le taux d'accroissement de f entre a et a+h est :}
\pause
\[\textcolor{red}{ \cfrac{f(a+h) - f(a)}{h} }\]
\pause
\end{exampleblock}
\begin{exampleblock}{Le nombre dérivé de f en a est : }
\pause
\[\textcolor{red}{f'(a) = \lim\limits_{h \rightarrow 0}\dfrac{f(a+h) - f(a)}{h}} \]
\pause
\end{exampleblock}
\begin{exampleblock}{L'équation d'une tangente à une courbe :}
\pause
\[\textcolor{red}{y = f'(a)(x-a)+f(a)} \]
\pause

\end{exampleblock}
\end{frame}

\end{document}

