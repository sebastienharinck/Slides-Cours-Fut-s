% ==================================================
% 1S - DERIVATION - COURS - UTILISATION DES DERIVEES
% ==================================================

% ADD BEAMER THEME
\documentclass[t]{beamer}

% ADD PACKAGES
\usepackage{etex}
\usepackage[english]{babel}
\usepackage[utf8]{inputenc}
\usepackage{pstricks-add}
\usepackage{pst-eucl}

\beamertemplatenavigationsymbolsempty

% DEFINE COLORS
\definecolor{red}{HTML}{DD0000}
\definecolor{green}{RGB}{51,110,23}
\setbeamercolor{block title example}{fg=green}

% DOCUMENT
\begin{document}

	\begin{frame}[label=pagebanale]
		\frametitle{Fonction Dérivée :}
		\pause
		\begin{enumerate}[a)]
			\item<+-> Le tableau de signe de la dérivée permet de déduire le tableau de variations de la fonction
			\item<+-> La dérivée sert aussi à déterminer l'extremum local d'une fonction
		\end{enumerate}
	\end{frame}

	\begin{frame}[label=pagebanale]
		\frametitle{a) Le tableau de signe de la dérivée permet de déduire le tableau de variations de la fonction : }
		\pause
		\begin{block}{Définition :}
		\pause
		Soit f une fonction définie et dérivable sur un intervalle I. \pause
		\begin{enumerate}[]
			\item<+-> Si f' est \textcolor{red}{positive} sur I, \pause alors f est \textcolor{red}{croissante} sur I. \pause
			\item<+-> Si f' est \textcolor{red}{négative} sur I, \pause alors f est \textcolor{red}{décroissante} sur I. \pause
			\item<+-> Si f' est \textcolor{red}{nulle} sur I, \pause alors f est \textcolor{red}{constante} sur I.
		\end{enumerate}
		\pause
		\end{block}
	\end{frame}

	\begin{frame}[label=pagebanale]
		\frametitle{b) Exemple}
		\begin{block}{Exemples : }
			Dresser le tableau de variations de la fonction de : $f(x) = 4^2 - 6x + 3$. 
		\end{block}
		\begin{block}{Solutions }
			f est une fonction polynôme, elle est donc dérivable sur $\mathbb{R}$. $f'(x) = 2 \times 4 x - 6 + 0 = 8x - 6$.
			
			\pause
			Il suffit de trouver le signe de f' pour en déduire les variations de f. On cherche lorsque $ 8x - 6 sup 0$ par exemple.
			$ 8x - 6 sup 0 consequence 8x  sup 6 consequence x sup \cfrac{8}{6} consequence \cfrac{4}{3}$ Donc f'(x) sera positive lorsque x sup à 4/3. On en déduit donc que f sera croissante lorsque x sup à 4/3. 
		\end{block}
	\end{frame}

	\begin{frame}[label=pagebanale]
		\frametitle{c) Les 5 règles de dérivation}
		\pause
		Soit u et v deux fonctions.
		\begin{tabular}{|c|c|c|}
			\hline
				\textbf{forme de f(x)} & \textbf{f'(x)} \\
			\hline
				$u+v$ & $u' + v'$ \\
			\hline
				$\lambda u$ & $\lambda u'$ \\
			\hline
				$uv $ & $u'v +v'u$ \\
			\hline 
				$\frac{u}{v}$ & $\frac{u'v-v'u}{v^2}$ \\
			\hline
				$u^n$ & $n \times u' u^{(n-1)}$ \\
			\hline
		\end{tabular}
		\begin{block}{Exemple :}
			\begin{enumerate}
				\item<+-> $f(x) = x^2 + x$. \pause Comme $(x^2)' = 2x$ \pause et $(x)' = 1$. \pause On en déduit que $f'(x) = 2x + 1$. \pause
				\item<+-> $g(x) = 42 \sqrt{x}$. \pause Dans notre cas on remarque que g(x) est de la forme $\lambda u$ \pause où $\lambda = 42$ \pause et $u = \sqrt{x}$. \pause Comme $u' = (\sqrt{x})' = \pause \frac{1}{2\sqrt{x}}$, \pause on en déduit que $g'(x) = \pause 42 \pause \times \pause \frac{1}{2\sqrt{x}} \pause = \frac{42}{2\sqrt{x}} \pause $
			\end{enumerate}
		\end{block}
	\end{frame}

	\begin{frame}
		\frametitle{Exemple avec (uv)' = u'v + v'u}
		\pause
		\begin{block}{Dériver h(x)}
			$h(x) = x^3 \sqrt{x}$. \pause Il s'agit d'une fonction de la forme uv \pause où u et v sont deux fonctions \pause telles que $u(x) = x^3$ \pause et $v(x) = \sqrt{x}$. \pause Nous allons utiliser la formule (uv)' = u'v + v'u. \pause Calculons :
			\begin{enumerate}{}
				\item<+-> \(u'(x)= 3 \times x^{(3-1)} = 3x^2 \)
				\item<+-> \(v'(x) = \frac{1}{2 \sqrt{x}} \) 
			\end{enumerate}
			\pause
			A partir d'ici, il suffit de remplacer : \pause
			\begin{enumerate}[]
				\item<+-> \(g'(x) = u'v + v'u = \pause 3x^2 \times \sqrt{x} + \frac{1}{2 \sqrt{x}} \times x^3\) \pause
				\item<+-> \(g'(x) = 3x^2 \sqrt{x} + \frac{x^3}{2 \sqrt{x}} \)
			\end{enumerate}	
		\end{block}
	\end{frame}

	\begin{frame}
		\frametitle{Exemple avec $\frac{u}{v}$ }
		\pause
		\begin{block}{Dériver i(x)}
			$i(x) = \frac{x^2 + x}{3x}$. \pause Il s'agit d'une fonction de la forme $\frac{u}{v}$ \pause où u et v sont deux fonctions \pause telles que $u(x) = x^2 + x$ \pause et $v(x) = 3x$. \pause Nous allons utiliser la formule $\left( \frac{u}{v} \right)' = \frac{u'v - v'u}{v^2}$. \pause Calculons :
			\begin{enumerate}{}
				\item<+-> \(u'(x)= 2x + 1 \)
				\item<+-> \(v'(x) = 3 \) 
			\end{enumerate}
			\pause
			A partir d'ici, il suffit de remplacer : \pause
			\begin{enumerate}[]
				\item<+-> \(i'(x) = \frac{u'v - v'u}{v^2} = \frac{(2x+1) \times 3x - 3 \times (x^2+1}{(3x)^2}\) \pause
				\item<+-> \(i'(x) = 3x^2 \sqrt{x} + \frac{x^3}{2 \sqrt{x}} \)
			\end{enumerate}	
		\end{block}
	\end{frame}
\end{document}