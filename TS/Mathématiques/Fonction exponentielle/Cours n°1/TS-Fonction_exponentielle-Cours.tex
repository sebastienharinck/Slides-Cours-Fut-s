\documentclass[t]{beamer}
%\usepackage[orientation=paysage,width=200,height=120,scale=2]{beamerposter}
\usepackage{etex}
\usepackage[english]{babel}
\usepackage[utf8]{inputenc}
\usepackage{pstricks-add}
\usepackage{pst-eucl}

\beamertemplatenavigationsymbolsempty
% \usetheme{Warsaw}

\definecolor{red}{HTML}{DD0000}
\definecolor{green}{RGB}{51,110,23}
\setbeamercolor{block title example}{fg=green}

\title{TS : Fonction Exponentielle : Cours}
\author{Sébastien Harinck}
\institute{www.cours-futes.com}
\date{}

\begin{document}
\begin{frame}
\titlepage
\end{frame}

\begin{frame}[label=pagebanale]
\frametitle{Courbe de la fonction exponentielle}
\pause
Les physiciens ont eu besoin de définir une fonction, afin de pouvoir faire des calculs.
\pause
GRAPHIQUE
\pause
D'après ce graphique, nous pouvons apprendre et retenir plusieurs informations capitales pour la suite.
\begin{enumerate}
\item<+-> la fonction f(x) est définie sur $\mathbb{R} $
\item<+-> f(0) = 1 
\item<+-> \( \lim\limits_{x \rightarrow +\infty} f(x) = +\infty \)
\item<+-> \( \lim\limits_{x \rightarrow -\infty} f(x) = 0 \)
\end{enumerate}
\end{frame}

\begin{frame}
\frametitle{Propriétés de la fonction exponentielle f(x) = exp(x)}
\pause
Il existe une unique fonction f dérivable sur $\mathbb{R}$ telle que f' = f et f(0) = 1.
\pause
\( exp(x)' = exp(x)\)
\pause
Pour tout réel x, $ exp(-x) = \dfrac{1}{exp(x)}$
\pause
Pour tout réel x, $ exp(x) \neq 0 $
\pause
Pour tout réel x, $ exp(x) > 0 $
\pause
Soient a et b $\in \mathbb{R}$,$ exp(a+b) = exp(a) \times exp(b)$
\end{frame}

\begin{frame}
\frametitle{Le nombre e}
On appelle e l'image de 1 par la fonction exp : e = exp(1).
\pause
\( e \approx 2.718\)
\end{frame}

\begin{frame}
\frametitle{Quelques relations :}
\pause
Pour tout réel x, on note $exp(x) = e^x$
\pause
\begin{enumerate}
\item<+-> \( e^0 = 1\)
\item<+-> \( e^1 = 1\) 
\item<+-> \( e^{a+b} = e^a \times e^b\) 
\item<+-> \( e^{na} = (e^a)^n\)
\item<+-> \( e^{-b} = \dfrac{1}{e^b}\)  
\item<+-> \( e^{a-b} = \dfrac{e^a}{e^b}\)  
\end{enumerate}
\end{frame}

\begin{frame}
\frametitle{Inception}
\pause
Par définition, la fonction exp est dérivable sur $ \mathbb{R}$ et a pour dérivée elle-même; comme elle est strictement positive, exp est strictement croissante sur $ \mathbb{R}$
\end{frame}

\begin{frame}
\frametitle{D'autres limites}
On dit que l'exponentielle l'emporte !
Croissance comparée des fonctions $x \rightarrow x$ et $x \rightarrow e^x$ :
\[ \lim\limits_{x \rightarrow + \infty} \dfrac{e^x}{x} = + \infty \]
et
\[ \lim\limits_{x \rightarrow - \infty} xe^x = 0\]
Une autre propriété qui découle de la définition du taux d'accroissement :
\[ \lim\limits_{x \rightarrow 0} \dfrac{e^x - 1}{x} = exp'(0) = 1\]
\end{frame}

\begin{frame}
\frametitle{Dérivée une fonction du type : $ e^{u(x)} $}
\pause
Soit une fonction u définie et dérivable sur un intervalle I. \\
La fonction f définie sur I par $f(x) = e^{u(x)}$ est dérivable sur I et a pour dérivée : 
\[f(x) = u'(x) \times e^{u(x)} \]
\end{frame}

\end{document}

