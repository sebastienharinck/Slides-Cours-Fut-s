\documentclass[t]{beamer}
%\usepackage[orientation=paysage,width=200,height=120,scale=2]{beamerposter}
\usepackage{etex}
\usepackage[english]{babel}
\usepackage[utf8]{inputenc}
\usepackage{pstricks-add}
\usepackage{pst-eucl}

\beamertemplatenavigationsymbolsempty
% \usetheme{Warsaw}

\definecolor{red}{HTML}{DD0000}
\definecolor{green}{RGB}{51,110,23}
\setbeamercolor{block title example}{fg=green}

\title{TS : Fonction Exponentielle : Exercice 1}
\author{Sébastien Harinck}
\institute{www.cours-futes.com}
\date{}

\begin{document}
\begin{frame}
\titlepage
\end{frame}

\begin{frame}
\frametitle{Lever des indéterminations de limites en $ + \infty $ et $ - \infty $}
\pause
Soit f la fonction définie sur $\mathbb{R}$ par $f(x) = e^x(5 - x)$.
\pause
Calculer la limite de f en $ + \infty $
\pause
Calculer la limite de f en $ - \infty $
\pause
\end{frame}

\begin{frame}
\frametitle{Résolution :}
\pause
La fonction est le produit des fonctions $x \rightarrow e^x$ et $x \rightarrow 5 - x$
\pause
\( \lim\limits_{x \rightarrow + \infty} e^x = + \infty\)
et
\( \lim\limits_{x \rightarrow + \infty} (5 -x) = - \infty\)
; par produit,
\( \lim\limits_{x \rightarrow + \infty} f(x) = - \infty	\).

Maintenant, calculons la limite en $ - \infty $
\pause
\( \lim\limits_{x \rightarrow - \infty} e^x = 0\)
et
\( \lim\limits_{x \rightarrow - \infty} 5 - x = + \infty \)
\pause
Forme Indéterminée $ \rightarrow $On ne peut pas conclure.
\pause
On développe : $ f(x) = 5e^x - xe^x$.
\pause
On a donc 
\( \lim\limits_{x \rightarrow - \infty} 5e^x = 0\)
et on utilise le résultat
\( \lim\limits_{x \rightarrow - \infty} xe^x = 0\)
\pause
Par somme, on obtient :
\pause
\[ \lim\limits_{x \rightarrow - \infty} f(x) = 0.\]
\end{frame}

\begin{frame}
\frametitle{Question 2 :}
\pause
Résoudre dans $ \mathbb{R}$ l'équation $e^{2x+1} = e^{-0,5x+4}$.
\pause
Résoudre dans $ \mathbb{R}$ l'inéquation $e^{-x+3} \leq e^{2x+9}$.
\pause
\end{frame}

\begin{frame}
\frametitle{Solution commentée}
\pause
Pour résoudre des équations ou des inéquations du type $ e^{u(x)} = e^{v(x)}$ ou $ e^{u(x)} \leq e^{v(x)} $,
on utilise la propriété : quels que soient les réels a et b, $e^a$ et $e^b$ équivaut à a = b ; $e^a \leq e^b$ équivaut à $a \leq b$.
\pause
Résoudre $e^{2x+1} = e^{-0,5x+4}$ équivaut à résoudre : $2x+1 = -0.5x+4$
\pause
\( 2.5x = 3\)
\pause
\( x = \dfrac{3}{2,5} = 1,2\)
\pause
Résoudre $e^{-x+3} \leq e^{2x+9}$ équivaut à résoudre : $-x+3 \leq 2x+9$
\pause
\( -3x \leq 6\)
\pause
\( x \geq -2\)
\pause
\end{frame}

\end{document}

