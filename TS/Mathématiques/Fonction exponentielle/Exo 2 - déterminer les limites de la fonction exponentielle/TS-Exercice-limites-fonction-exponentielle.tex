\documentclass[t]{beamer}
%\usepackage[orientation=paysage,width=200,height=120,scale=2]{beamerposter}
\usepackage{etex}
\usepackage[english]{babel}
\usepackage[utf8]{inputenc}
\usepackage{pstricks-add}
\usepackage{pst-eucl}

\beamertemplatenavigationsymbolsempty
% \usetheme{Warsaw}

\definecolor{red}{HTML}{DD0000}
\definecolor{green}{RGB}{51,110,23}
\setbeamercolor{block title example}{fg=green}

\title{TS : Fonction Exponentielle : Exercice 2}
\author{Sébastien Harinck}
\institute{www.cours-futes.com}
\date{}

\begin{document}
\begin{frame}
\titlepage
\end{frame}

\begin{frame}
\frametitle{Déterminer les limites en $- \infty $ et en $+\infty$ des fonctions suivantes :}
\pause
\begin{enumerate}
\item<+-> f définie sur $\mathbb{R}$ par $ f(x) = e^x +2x+1$
\item<+-> h définie sur $\mathbb{R}$ par $ h(x) = e^x-x^2+2x+9$
\item<+-> g définie sur $\mathbb{R}$ par $ g(x) = e^x(2x+1)$
\item<+-> j définie sur $\mathbb{R}$ par $ j(x) = e^{-x}(-x+1)$
\item<+-> i définie sur $\mathbb{R}$ par $ i(x) = \dfrac{e^x-x}{e^x+1}$
\item<+-> k définie sur $\mathbb{R}$ par $ k(x) = xe^x+x^2+x-4$
\end{enumerate}
\end{frame}

\begin{frame}
\frametitle{f définie sur $\mathbb{R}$ par $ f(x) = e^x +2x+1$}
\pause
en $- \infty $ : \\
\( \lim\limits_{h \rightarrow -\infty} e^x = 0 \) \\
\( \lim\limits_{h \rightarrow -\infty} (2x+1) = -\infty \) \\
Par somme, nous obtenons : $ \lim\limits_{h \rightarrow -\infty}f(x)= -\infty$
en $+ \infty $ : \\
\( \lim\limits_{h \rightarrow +\infty} e^x = + \infty \) \\
\( \lim\limits_{h \rightarrow +\infty} (2x+1) = +\infty \) \\
Par somme, nous obtenons : $ \lim\limits_{h \rightarrow +\infty}f(x)= + \infty$
\end{frame}

\begin{frame}
\frametitle{h définie sur $\mathbb{R}$ par $ h(x) = e^x-x^2+2x+9$}
\pause
en $- \infty $ : \\
\( \lim\limits_{h \rightarrow -\infty} e^x = 0 \) \\
\( \lim\limits_{h \rightarrow -\infty} (-x^2+2x+9) = (x(-x+2) + 9) -\infty \) \\
Par somme, nous obtenons : $ \lim\limits_{h \rightarrow -\infty}f(x)= -\infty$
\end{frame}
\end{document}

