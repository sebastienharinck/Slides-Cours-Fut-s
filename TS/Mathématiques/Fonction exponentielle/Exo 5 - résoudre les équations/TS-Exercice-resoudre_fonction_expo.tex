\documentclass[t]{beamer}
%\usepackage[orientation=paysage,width=200,height=120,scale=2]{beamerposter}
\usepackage{etex}
\usepackage[english]{babel}
\usepackage[utf8]{inputenc}
\usepackage{pstricks-add}
\usepackage{pst-eucl}

\beamertemplatenavigationsymbolsempty
% \usetheme{Warsaw}

\definecolor{red}{HTML}{DD0000}
\definecolor{green}{RGB}{51,110,23}
\setbeamercolor{block title example}{fg=green}

\title{TS : Fonction Exponentielle : Exercice 2}
\author{Sébastien Harinck}
\institute{www.cours-futes.com}
\date{}

\begin{document}
\begin{frame}
\titlepage
\end{frame}

\begin{frame}
\frametitle{Résoudre dans $ \mathbb{R} $ les équations suivantes :}
\pause
\begin{enumerate}[1)]
\item<+-> \( e^{x} -1 = 0\)
\item<+-> \( e^{x} +1 = 0\)
\item<+-> \( e^{2x} = e^{x+4}\)
\end{enumerate}
\end{frame}

\begin{frame}
\frametitle{1) $e^{x} -1 = 0$}
\begin{enumerate}[]
\item<+-> \( e^{x} -1 = 0 \)
\item<+-> \( e^{x} = 1 \)
\item<+-> \( x = 0 \)
\end{enumerate}
\pause
Justification :
\pause
\begin{enumerate}
\item Vous devez savoir que $e^0 = 1$. \pause Il s'agit d'une propriété de la fonction exponentielle. 
\pause
\item Il n'y a pas d'autres solutions dans $ \mathbb{R} $. \pause Comme la fonction exponentielle est strictement croissante sur $ \mathbb{R} $, \pause on en déduit que l'équation $e^x = m$ admet une unique solution dans $ \mathbb{R} $.
\end{enumerate}
\pause
Explication Graphique :
...
\pause
\end{frame}

\begin{frame}
\frametitle{2) $e^{x} +1 = 0$}
\begin{enumerate}[]
\item<+-> \( e^{x} +1 = 0 \)
\item<+-> \( e^{x} = -1 \)
\end{enumerate}
Comme la fonction exponentielle est toujours positive sur $ \mathbb{R} $, il n'existe aucune solution pour résoudre cette équation dans $\mathbb{R}$.
\end{frame}

\begin{frame}
\frametitle{3) $e^{2x} = e^{x+4}$}
\pause
Une des propriétés de l'exponentielle est que :
\( e^a = e^b \Leftrightarrow a = b\)
\pause
donc :
\pause
\begin{enumerate}[$\Leftrightarrow$]
\item<+-> \( e^{2x} = e^{x+4} \)
\item<+-> \(2x = x+4 \)
\item<+-> \( x = 4 \)
\end{enumerate}
\end{frame}

\end{document}