\documentclass[t]{beamer}
%\usepackage[orientation=paysage,width=200,height=120,scale=2]{beamerposter}
\usepackage{etex}
\usepackage[english]{babel}
\usepackage[utf8]{inputenc}
\usepackage{pstricks-add}
\usepackage{pst-eucl}

\beamertemplatenavigationsymbolsempty
% \usetheme{Warsaw}

\definecolor{red}{HTML}{DD0000}
\definecolor{green}{RGB}{51,110,23}
\setbeamercolor{block title example}{fg=green}

\title{TS : Fonction Exponentielle : Exercice 2}
\author{Sébastien Harinck}
\institute{www.cours-futes.com}
\date{}

\begin{document}
\begin{frame}
\titlepage
\end{frame}

\begin{frame}
\frametitle{Exercice}
\pause
Soit f la fonction définie sur $ \mathbb{R} $ par :
\pause
\( f(x) = 2x-1-2e^x \) \\
\pause
1.a) Déterminer les limites de f en $ - \infty $ et en $ + \infty $. \\
\pause
Démontrer que la droite d'équation $ y = 2x-1 $ est asymptote à la courbe C représentant f en $ - \infty $. \\
\pause
Calculer la dérivée de f et dresser le tableau de variations de f. 
\end{frame}

\end{document}