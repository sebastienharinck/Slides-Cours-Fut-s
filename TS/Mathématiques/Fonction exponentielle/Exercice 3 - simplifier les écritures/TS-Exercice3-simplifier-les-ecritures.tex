\documentclass[t]{beamer}
%\usepackage[orientation=paysage,width=200,height=120,scale=2]{beamerposter}
\usepackage{etex}
\usepackage[english]{babel}
\usepackage[utf8]{inputenc}
\usepackage{pstricks-add}
\usepackage{pst-eucl}

\beamertemplatenavigationsymbolsempty
% \usetheme{Warsaw}

\definecolor{red}{HTML}{DD0000}
\definecolor{green}{RGB}{51,110,23}
\setbeamercolor{block title example}{fg=green}

\title{TS : Fonction Exponentielle : Exercice 3}
\author{Sébastien Harinck}
\institute{www.cours-futes.com}
\date{}

\begin{document}
\begin{frame}
\titlepage
\end{frame}

\begin{frame}
\frametitle{Simplifier les écritures suivantes :}
\pause
\begin{enumerate}
\item<+-> \( \cfrac{e^4 \times e^3}{e^{12} \times e}\)
\item<+-> \( \cfrac{e^{3x} \times e^{-1}}{e^{2x+2} \times e^{-2}}\)
\item<+-> \( \cfrac{e^{-1} \times e^3}{e^{-x} \times e} \times e^2\)
\item<+-> \( \cfrac{e^{\cfrac{1}{x}} \times (e^{2x})^3 }{e^{x}}\)
\end{enumerate}
\end{frame}

\begin{frame}
\frametitle{1) \( \cfrac{e^4 \times e^3}{e^{12} \times e} \)}
\pause
Nous allons utiliser les propriétés suivantes :
\pause
\begin{enumerate}
\item<+-> \( e^1 = e \)
\item<+-> \( \textcolor{red}{ e^{a+b} = e^ae^b } \)
\item<+-> \( \textcolor{red}{ e^{a-b} = \dfrac{e^a}{e^b} } \)
\end{enumerate}
\pause
\( \cfrac{e^4 \times e^3}{e^{12} \times e} \pause = \cfrac{e^{4+3}}{e^{12} \times e^1} \pause =\)
\( \cfrac{e^{4+3}}{e^{12+1}} \pause = \cfrac{e^7}{e^{13}} \pause = \)
\( e^{7-13} \pause = e^{-6} \cong \)
\end{frame}

\begin{frame}
\frametitle{2) \( \cfrac{e^{3x} \times e^{-1}}{e^{2x+2} \times e^{-2}}\)}
\pause
Nous allons utiliser les propriétés suivantes :
\pause
\begin{enumerate}
\item<+-> \( e^1 = e \)
\item<+-> \( \textcolor{red}{ e^{a+b} = e^ae^b } \)
\item<+-> \( \textcolor{red}{ e^{a-b} = \dfrac{e^a}{e^b} } \)
\end{enumerate}
\pause
\( \cfrac{e^{3x} \times e^{-1}}{e^{2x+2} \times e^{-2}} = \cfrac{e^{3x + (-1)}}{e^{2x+2 + (-2)}} =\)
\( \cfrac{e^{3x -1}}{e^{2x}} = e^{3x-1-2x} = e^{x-1} \)
\end{frame}

\begin{frame}
\frametitle{3) \( \cfrac{e^{-1} \times e^3}{e^{-x} \times e} \times e^2 \) }
\pause
Nous allons utiliser les propriétés suivantes :
\pause
\begin{enumerate}
\item<+-> \( e^1 = e \)
\item<+-> \( \textcolor{red}{ e^{a+b} = e^ae^b } \)
\item<+-> \( \textcolor{red}{ e^{a-b} = \dfrac{e^a}{e^b} } \)
\end{enumerate}
\pause
\( \cfrac{e^{-1} \times e^3}{e^{-x} \times e} \times e^2 = \cfrac{e^{-1+3}}{e^{-x+1}} \times e^2 = \)
\( \cfrac{e^2}{e^{-x+1}} \times e^2 = \cfrac{e^2 \times e^2	}{e^{-x+1}} \)
\( = \cfrac{e^{2+2}}{e^{-x+1}} = \cfrac{e^4}{e^{-x+1}} = \)
\( e^{4 - \textcolor{red}{ ( }-x + 1 \textcolor{red}{ ) }} = e^{4 + x - 1} = e^{3+x} \)
\end{frame}

\begin{frame}
\frametitle{4) \( \cfrac{e^{\cfrac{1}{x}} \times (e^{2x})^3 }{e^{x}}\) }
\pause
Nous allons utiliser les propriétés suivantes :
\pause
\begin{enumerate}
\item<+-> \( e^1 = e \)
\item<+-> \( \textcolor{red}{ e^{a+b} = e^ae^b } \)
\item<+-> \( \textcolor{red}{ e^{a-b} = \dfrac{e^a}{e^b} } \)
\item<+-> \( e^{na} = (e^a)^n, \) pour tout n appartenant à Z
\end{enumerate}
\pause
\( \cfrac{e^{\cfrac{1}{x}} \times (e^{2x})^3 }{e^{x}} = \cfrac{e^{\cfrac{1}{x}} \times e^{3 \times 2x}}{e^{x}} = \)
\pause
\( \cfrac{e^{\cfrac{1}{x}} \times e^{6x}}{e^{x}} = \cfrac{e^{\cfrac{1}{x}+6x}}{e^{x}} = \)
\pause
\( \cfrac{e^{\cfrac{1+6x^2}{x}}}{e^{x}} = e^{\dfrac{1+6x^2}{x}-x} = e^{\dfrac{1+6x^2-x^2}{x}} = e^{\dfrac{1+5x^2}{x}} \)
\end{frame}

\end{document}

