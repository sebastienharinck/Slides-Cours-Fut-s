\documentclass[t]{beamer}
%\usepackage[orientation=paysage,width=200,height=120,scale=2]{beamerposter}
\usepackage{etex}
\usepackage[english]{babel}
\usepackage[utf8]{inputenc}
\usepackage{pstricks-add}
\usepackage{pst-eucl}

\beamertemplatenavigationsymbolsempty
% \usetheme{Warsaw}

\definecolor{red}{HTML}{DD0000}
\definecolor{green}{RGB}{51,110,23}
\setbeamercolor{block title example}{fg=green}

\title{TS : Fonction Exponentielle : Exercice 3}
\author{Sébastien Harinck}
\institute{www.cours-futes.com}
\date{}

\begin{document}
\begin{frame}
\titlepage
\end{frame}

\begin{frame}
\frametitle{Dériver les fonctions suivantes :}
\begin{enumerate}
\item<+-> f définie sur $\mathbb{R} $ par $f(x) = xe^x$
\item<+-> g définie sur $\mathbb{R} - \lbrace \dfrac{1}{2}\rbrace $ par $g(x) = \dfrac{e^x+1}{2x+1}$
\item<+-> h définie sur $\mathbb{R} $ par $h(x) = (3x^2-2x+1)e^{x}$
\item<+-> i définie sur $\mathbb{R} $ par $i(x) = (x+e^x)^4$
\end{enumerate}
\end{frame}

\begin{frame}
\frametitle{f définie sur $\mathbb{R} $ par $f(x) = xe^x$}
\pause
Il s'agit d'une fonction de la forme $u \times v$. \\
\pause
\( (uv)' = u'v + v'u \)
\pause  
\begin{enumerate}
\item<+-> \( u = x \)
\item<+-> \( u'= 1 \)
\item<+-> \( v = e^x\)
\item<+-> \( v'= e^x \)
\end{enumerate}
\pause
\( f'(x) = u'v + v'u = 1 \times e^x + e^x \times x = e^x + e^xx = e^x (1+x)  \)
\end{frame}

\begin{frame}
\frametitle{g définie sur $\mathbb{R} - \lbrace \dfrac{1}{2}\rbrace $ par $g(x) = \dfrac{e^x+1}{2x+1}$}
\pause
Il s'agit d'une fonction de la forme $ \dfrac{u}{v}$ \\
\pause
\( (\dfrac{u}{v})' = \dfrac{u'v-v'u}{v^2} \)
\pause
\begin{enumerate}
\item<+-> \(u =  e^x+1 \)
\item<+-> \(u'= e^x \)
\item<+-> \(v = 2x+1 \)
\item<+-> \(v' = 2 \)
\end{enumerate}
\pause
\( g'(x)=  \dfrac{u'v-v'u}{v^2} = \dfrac{e^x(2x+1) - 2(e^x+1)}{(2x+1)^2}\) \\
\( g'(x) = \dfrac{2xe^x+e^x - 2e^x-2}{(2x+1)^2}\) \\
\( g'(x) = \dfrac{2xe^x-e^x-2}{(2x+1)^2}\)
\end{frame}

\begin{frame}
\frametitle{h définie sur $\mathbb{R} $ par $h(x) = (3x^2-2x+1)e^x$}
\pause
Il s'agit d'une fonction de la forme $ u \times v $ 
\pause
\( (uv)' = u'v + v'u \)
\pause
\begin{enumerate}
\item<+-> \(u =  3x^2-2x+1 \)
\item<+-> \(u'= 6x - 2 \)
\item<+-> \(v = e^x \)
\item<+-> \(v' = e^x \)
\end{enumerate}
\( h'(x) = u'v + v'u = (6x - 2)e^x + e^x(3x^2-2x+1)\) \\
\pause
\( h'(x) = 6xe^x -2e^x + 3x^2e^x -2xe^x+e^x \) \\
\pause
\( h'(x) = 3x^2e^x + 4xe^x -e^x\)
\end{frame}



\end{document}

