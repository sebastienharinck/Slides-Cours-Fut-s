\documentclass[t]{beamer}
%\usepackage[pass,letterpaper]{geometry}
%\usepackage[orientation=paysage,width=200,height=120,scale=2]{beamerposter}
\usepackage{etex}
\usepackage[english]{babel}
\usepackage[utf8]{inputenc}
\usepackage{pstricks-add}
\usepackage{pst-eucl}

\beamertemplatenavigationsymbolsempty
% \usetheme{Warsaw}

\definecolor{red}{HTML}{DD0000}
\definecolor{green}{RGB}{51,110,23}
\setbeamercolor{block title example}{fg=green}

\title{TS : Fonction Exponentielle : Exercice 13}
\author{Sébastien Harinck}
\institute{www.cours-futes.com}
\date{}



\begin{document}
\begin{frame}
\titlepage
\end{frame}

\begin{frame}
\frametitle{Exercice type bac}
Soit f la fonction définie sur $\mathbb{R}$ par $f(x)= x + 1 - \dfrac{4e^x}{e^x+1}$
et C sa courbe représentatitve.
\begin{enumerate}
\item<+-> Déterminer la limite de f en $ - \infty $
\item<+-> Démontrer que la droite \textit{D} d'équation $y=x+1$ est asymptote à la courbe C en $ -\infty $
\item<+-> Etudier la position de C par rapport à D.
\item<+-> Déterminer la limite de f en $ + \infty $
\item<+-> Démontrer que la droite D' d'équation y = x - 3 est asymptote à la courbe C en $ en + \infty $
\item<+-> Etudier la position de C par rapport à D'
\item<+-> Calculer f'(x) et montrer que, pour tout réel x, $f'(x)  = \left( \dfrac{e^x-1}{e^x+1} \right)^2 $
\item<+-> Etudier les variations de f et dresser le tableau de variation de f.
\item<+-> Tracer la courbe et ses asymptotes 
\end{enumerate}
\end{frame}

\begin{frame}
\frametitle{1) Déterminer la limite de f en $ - \infty $}
\( f(x)= x + 1 - \dfrac{4e^x}{e^x+1} \)
\begin{enumerate}[]
\item \( \lim\limits_{ x \rightarrow -\infty } x + 1 = - \infty \)
\item \( \lim\limits_{ x \rightarrow - \infty } 4e^x = \pause 0 \)
\item \( \lim\limits_{ x \rightarrow - \infty } e^x + 1 = \pause 1 \)
\item D'où \( \lim\limits_{ x \rightarrow -\infty } \dfrac{4e^x}{e^x + 1} = \pause 0 \)
\item Par somme, \pause \( \lim\limits_{ x \rightarrow -\infty } f(x) = -\infty \)
\end{enumerate}
\end{frame}

\begin{frame}
\frametitle{Démontrer que la droite \textit{D} d'équation $y=x+1$ est asymptote à la courbe C en $ -\infty $}
\( f(x)= x + 1 - \dfrac{4e^x}{e^x+1} \)
\\
Théorème :
\\
Si, pour tout x de \pause 
$]- \infty ; c]$
, f peut s'écrire sous la forme
\pause
$f(x) = ax+b + g(x)$,
\pause
avec a non nul et 
\pause
\( \lim\limits_{ x \rightarrow - \infty } g(x) = 0 \),
\pause
alors la droite d'équation $ y = ax+b $ est asymptote oblique à Cf en $ - \infty $
\\
Résolution :
\\
La fonction f est écrite sous la forme $ f(x) = x +1 - \dfrac{4e^x}{e^x+1}$ avec 
\( \lim\limits_{ x \rightarrow - \infty } \dfrac{4e^x}{e^x+1} = 0 \)
\\
\pause
Donc la droite d'équation $y=x+1$ est asymptote à C en $ -\infty $ 
\end{frame}

\begin{frame}
\frametitle{3) Etudier la position de C par rapport à D.}
Pour étudier la position de C par rapport à D, on étudie le signe de la différence f(x) - (x+1).
\\
\pause
Soit \( f(x)-(x-1)= - \dfrac{4e^x}{e^x+1} \)
\\
\pause
Or nous savons que $ \forall x, \pause e^x > 0$,
\pause
Donc \( - \dfrac{4e^x}{e^x+1} < 0 \)
\\
\pause
Par conséquent, \pause
la courbe C est toujours située sous la droite D.

\end{frame}

\begin{frame}
\frametitle{Déterminer la limite de f en $ + \infty $}
\( f(x)= x + 1 - \dfrac{4e^x}{e^x+1} \)
\\
\pause
FORME INDETERMINEE -> Transformer l'écriture
\\
\pause
\( x + 1 - \dfrac{4e^x}{e^x+1} = \pause x + 1 - \pause \dfrac{4e^x}{e^x(1+\dfrac{1}{e^x})} = \pause \dfrac{4e^x}{e^x(1+e^{-x})} \)
\\
\pause
Rmq : \( e^x \times e^{-x} = \pause e^x \times \dfrac{1}{e^x} = \pause 1\)
\\
\pause
Nous pouvons désormais simplifier :
\\
\pause
\( x + 1 - \dfrac{4e^x}{e^x(1+e^{-x})} = x + 1 - \dfrac{4}{1+e^{-x}} \)
\\
\pause
\begin{enumerate}[]
\item \( \lim\limits_{ x \rightarrow + \infty } x + 1 = \pause + \infty \) 
\pause
\item \( \lim\limits_{ x \rightarrow + \infty } e^{-x} = \pause 0 \) \pause donc \( \lim\limits_{ x \rightarrow + \infty } \dfrac{4}{1+e^{-x}} = 4 \)
\pause
\end{enumerate}
Par somme, \pause on en déduit que :
\[ \lim\limits_{ x \rightarrow + \infty } f(x) = + \infty \]
\end{frame}




\end{document}

