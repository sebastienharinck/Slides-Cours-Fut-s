\documentclass[t]{beamer}
%\usepackage[orientation=paysage,width=200,height=120,scale=2]{beamerposter}
\usepackage{etex}
\usepackage[english]{babel}
\usepackage[utf8]{inputenc}
\usepackage{pstricks-add}
\usepackage{pst-eucl}

\beamertemplatenavigationsymbolsempty
% \usetheme{Warsaw}

\definecolor{red}{HTML}{DD0000}
\definecolor{green}{RGB}{51,110,23}
\setbeamercolor{block title example}{fg=green}

\title{TS : Fonction Exponentielle : Exercice 15}
\author{Sébastien Harinck}
\institute{www.cours-futes.com}
\date{}



\begin{document}
\begin{frame}
\titlepage
\end{frame}

\lim\limits_{ x \rightarrow 0 }

\begin{frame}
\frametitle{Résoudre l'exercice suivant :}
Soit f la fonction définie sur $ \mathbb{R} $ par :
\pause
\[ f(x) = e^x + 2e^{-x} + x \]
\begin{enumerate}
\item<+-> Etudier les variations de f sur $ \mathbb{R} $
\item<+-> Déterminer les limites de f en $ - \infty $ et en $ + \infty $
\item<+-> Dresser le tableau de variations de f
\end{enumerate}
\end{frame}

\begin{frame}
\frametitle{Etudier les variations de f sur $ \mathbb{R} $}
\pause
\( f(x) = e^x + 2e^{-x} + x \)
\\
Pour étudier les variations d'une fonction, il est conseillé de calculer sa dérivée et d'étudier son signe.
\\
\pause
\( f'(x) =  e^x \textcolor{red}{-} 2e^{-x} + 1 = e^x + 2 \dfrac{1}{e^x} + 1 \)
\\
\pause
Bien que la fonction exponentielle soit toujours positive sur $ \mathbb{R} $, on ne peut pas déterminer tout de suite le signe de la fonction f'(x).
\\
\pause
Dans cet exercice, nous allons utiliser une technique difficile (mais que vous devez connaître :) )
\\
\pause
Le but est d'exprimer la fonction avec X où $ X =  e ^x $.
\\
\pause
Comme ceci par exemple : 
\( 3X^2 - \dfrac{7}{3}X - 2 \)
\\
\pause

\end{frame}

\begin{frame}
\frametitle{Etudier les variations de f sur $ \mathbb{R} $}
\pause
On va commencer par mettre la fonction sous un même dénomitateur.
\\
\pause
\( f'(x) = e^x + 2 \dfrac{1}{e^x} + 1  = \pause \dfrac{e^{2x} - 2 + e^x}{e^x} = \dfrac{e^{2x} + e^x - 2}{e^x} \)
\\
\pause
Si on pose $ X = e^x $, \pause on obtient :
\pause
\( f'(x) = \dfrac{X^2 + X - 2}{X} \)
\\
\pause
\textcolor{red}{Pourquoi $ e^{2x} = X^2 $ ?}
\pause
Parce que $ e^{2x} = (e^x)^2 $
\\
\pause
Si vous ne pensez qu'en terme de X, vous êtes capable de déterminer le signe de $\dfrac{X^2 + X - 2}{X}$ sur ...
\\
\pause
\textcolor{green}{Remarque : Lorsque que l'on parle du signe ou de la variation d'une fonction c'est toujours sur un INTERVALLE}

\end{frame}

\end{document}