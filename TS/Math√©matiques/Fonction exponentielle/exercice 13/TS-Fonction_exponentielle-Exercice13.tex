\documentclass[t]{beamer}
%\usepackage[orientation=paysage,width=200,height=120,scale=2]{beamerposter}
\usepackage{etex}
\usepackage[english]{babel}
\usepackage[utf8]{inputenc}
\usepackage{pstricks-add}
\usepackage{pst-eucl}

\beamertemplatenavigationsymbolsempty
% \usetheme{Warsaw}

\definecolor{red}{HTML}{DD0000}
\definecolor{green}{RGB}{51,110,23}
\setbeamercolor{block title example}{fg=green}

\title{TS : Fonction Exponentielle : Exercice 13}
\author{Sébastien Harinck}
\institute{www.cours-futes.com}
\date{}

\begin{document}
\begin{frame}
\titlepage
\end{frame}

\begin{frame}
\frametitle{Résoudre l'exercice suivant}
Soit f la fonction définie sur $ ]0; + \infty [$ par $f(x)= e^{\dfrac{1}{x}} $ 
\pause
et C sa courbe représentative.
\begin{enumerate}
\item<+-> Déterminer les limites de f en 0 et en $ + \infty $.
\item<+-> Calculer la dérivée de f et dresser le tableau de variations de f.
\end{enumerate} 
\end{frame}

\begin{frame}
\frametitle{Les limites de $f(x)= e^{\dfrac{1}{x}} $ }
La fonction f est une fonction de la forme $e^u$ où
\( u(x) = \dfrac{1}{x} \)
\pause
\\

On utilise le théorème sur la limite d'un fonction composée :
\\
\pause
Si
\( \lim\limits_{x \rightarrow a} f(x) = b \)
et 
\( \lim\limits_{x \rightarrow b} g(x) = c \)
alors 
\( \lim\limits_{x \rightarrow a} g(f(x)) = c \)

\pause
Nous allons appliquer ce théorème :
\pause

\( \lim\limits_{x \rightarrow 0+} \dfrac{1}{x} = \textcolor{red}{+ \infty} \)
et
\( \lim\limits_{x \rightarrow \textcolor{red}{+\infty}} e^x = + \infty \)
alors
\( \lim\limits_{x \rightarrow 0+} f(x) = + \infty \)
\end{frame}

\begin{frame}
\frametitle{Les limites de $f(x)= e^{1/x} $ }
\pause
Vous avez sûrement dû comprendre la logique.
\pause
Essayez d'appliquer le théorème mais en $ + \infty $.
\pause

\( \lim\limits_{x \rightarrow + \infty} \dfrac{1}{x} = \textcolor{red}{0} \)
et
\pause
\( \lim\limits_{x \rightarrow \textcolor{red}{0}} e^x = 1 \)
donc
\pause
\( \lim\limits_{x \rightarrow + \infty} f(x) = 1\)
\end{frame}

\begin{frame}
\frametitle{Calculer la dérivée de f et dresser le tableau de variations de f.}
\pause
Comme annoncé précédemment, f est de la forme : $ e^u $.
\pause
Un de nos théorèmes est que la dérivée 
\( (e^u)' = u'e^u \)
\\
\pause

\( u' = \pause \left( \dfrac{1}{x} \right)' \pause = \dfrac{-1}{x^2} \)
\pause
\\
On en déduit que 
\( f'(x) = \dfrac{-1}{x^2} e^{\dfrac{1}{x}}\)
\\

\pause
Comme $x^2$ est toujours positive sur $ ] 0 ; + \infty [ $.
\pause
Donc 
\( \dfrac{-1}{x^2} \)
est toujours négative sur $ ] 0 ; + \infty [ $.
\\
\pause
et comme 
\( e^{\dfrac{1}{x}} \)
sera toujours positive sur $ ] 0 ; + \infty [ $.
\\
\pause
car l'exponentielle est toujours positive sur $ \mathbb{R} $ 
\\
\pause
On déduit que la fonction f'(x) sera toujours négative sur $ ] 0 ; + \infty [ $ et que f(x) sera toujours décroissante sur ce même intervalle. 
 


\end{frame}


\end{document}

