\documentclass[t]{beamer}
%\usepackage[orientation=paysage,width=200,height=120,scale=2]{beamerposter}
\usepackage{etex}
\usepackage[english]{babel}
\usepackage[utf8]{inputenc}
\usepackage{pstricks-add}
\usepackage{pst-eucl}

\beamertemplatenavigationsymbolsempty
% \usetheme{Warsaw}

\definecolor{red}{HTML}{DD0000}
\definecolor{green}{RGB}{51,110,23}
\setbeamercolor{block title example}{fg=green}

\title{TS : Fonction Exponentielle : Exercice 10}
\author{Sébastien Harinck}
\institute{www.cours-futes.com}
\date{}

\begin{document}
\begin{frame}
\titlepage
\end{frame}

\begin{frame}
\frametitle{Exercice 10 :}
\pause
Soit f la fonction définie par $ f(x) = \dfrac{x}{e^x-1} $ et C sa courbe représentative dans un repère du plan. \\
\pause
\begin{enumerate}
\item<+-> Quel est l'ensemble de définition de f ?
\item<+-> Déterminer les limites de f en 0, $ - \infty $ et en $ + \infty $.
\item<+-> Donner les asymptotes à la courbe C
\end{enumerate}
\end{frame}

\end{document}